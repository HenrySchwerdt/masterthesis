\chapter*{Zusammenfassung}
\label{cha:zusammenfassung}


Diese Masterarbeit präsentiert eine innovative Web-Erweiterung, die die Privatsphäre der Nutzer effektiv stärkt,
indem sie Tracking-Anfragen in Echtzeit blockiert. Mithilfe von Machine-Learning-Modellen zeigt die Erweiterung
bemerkenswerte Leistungen bei der Identifizierung und Abwehr von Web-Tracking-Anfragen. Zwei Modelle, nämlich Modell 1
und Modell 2, wurden anhand des Datensatzes der Top 1.000 Websites von Tranco evaluiert. Modell 1 erreichte eine
beeindruckende Präzision von 0,98 und eine Rückrufquote von 0,99 für nicht verfolgende Elemente sowie eine Präzision
von 0,97, eine Rückrufquote von 0,96 und einen F1-Score von 0,96 für verfolgende Elemente. Ebenso erzielte Modell 2
herausragende Präzisions-, Rückruf- und F1-Scores von 0,99, 0,98 und 0,98 für nicht verfolgende Elemente sowie 0,95, 0,97
und 0,96 für verfolgende Elemente.

Die Echtzeitfunktion der Web-Erweiterung gewährleistet sofortiges Handeln beim Erkennen von Tracking-Anfragen. Durch
das zeitnahe Eingreifen und Blockieren von Trackern wird die Privatsphäre der Nutzer erheblich gestärkt und die Risiken
im Zusammenhang mit Datenverfolgung und Online-Tracking minimiert. Dieser dynamische Ansatz ermöglicht die Verwendung
von ML-Modellen in der Web-Tracking-Umgebung, die bisher von manuell erstellten Sperrlisten geprägt war. Die Leistung
der Modelle in Echtzeit zeigt Ergebnisse, die denen des Anzeigen- und Tracking-Blockers des Brave-Browsers nahekommen.

Um eine reibungslose Implementierung und kontinuierliche Verbesserung zu ermöglichen, wurde eine robuste Backend-Infrastruktur
entwickelt. Diese Infrastruktur ermöglicht eine nahtlose Integration neuer Machine-Learning-Modelle in die Web-Erweiterung
und gewährleistet ihre Anpassungsfähigkeit an sich entwickelnde Tracking-Techniken, um die Wirksamkeit der Erweiterung
im Laufe der Zeit zu steigern. Durch die Nutzung dieser Backend-Infrastruktur bleibt die Web-Erweiterung stets aktuell
und kann aufkommenden Datenschutzbedrohungen effektiv begegnen, um den Nutzern anhaltenden Schutz zu bieten.

Durch eine sorgfältige Bewertung und umfangreiche Tests hat die Web-Erweiterung ihre Fähigkeit zur effektiven Blockierung
von Tracking-Anfragen erfolgreich unter Beweis gestellt. Die Einbindung von Machine-Learning-Modellen ermöglicht eine
präzise Identifizierung und Abwehr von Trackern. Die Echtzeitfunktion der Erweiterung gewährleistet eine umgehende Reaktion
und verhindert unbefugte Datensammlung, um die Privatsphäre der Nutzer zu stärken. Mit einer skalierbaren Backend-Infrastruktur
bleibt die Web-Erweiterung agil und flexibel, um sich kontinuierlich weiterentwickelnden Tracking-Methoden entgegenzustellen.
Diese Masterarbeit leistet einen Beitrag zur Verbesserung des Datenschutzes in der digitalen Welt und bietet den Nutzern eine
auf ML basierende Lösung, um Online-Tracking zu bekämpfen und ihre persönlichen Informationen zu schützen.
