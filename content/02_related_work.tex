\chapter{Related Work}
\label{cha:relatedwork}

Web tracking is a pervasive phenomenon on the internet, where users are 
often unaware of the extent to which their online activities are monitored
and recorded by third-party trackers. The use of tracking technologies such
as cookies, pixels, and fingerprinting can allow trackers to collect large 
amounts of personal information about users, including their browsing history,
demographics, and online behavior, without their explicit consent or knowledge 
\cite{englehardt2016online,hoofnagle2010different}. This can lead to a range of
negative consequences for users, including invasion of privacy, identity theft,
and discrimination \cite{hannak2013measuring,narayanan2010myths}.

With increasing presence of web trackers on the Internet, the need for effective
web tracker blocking tools has become more apparent. This need has been recognized
by both the public and various organizations, as evidenced by the popularity of ad
blockers and security browsers such as \textit{Brave}, as well as secure search engines
like \textit{StartPage}. These tools provide users with a greater control over their
online privacy and security, which is especially important in today's age of increasing
digital surveillance and cyber threats. Furthermore, researchers continue to explore
new techniques for detecting and blocking web trackers, which underscores the importance
of this field of research.

\section{Severity of web tracking} 

Web tracking has become a crucial issue in recent years, as it poses 
a serious threat to user privacy on the Internet. Englehardt et al. \cite{englehardt2016online}
conducted a comprehensive study on web tracking by crawling the top one
million websites and analyzing the presence of third-party trackers.
The study revealed that web tracking is pervasive, with more than 80\%
of the top websites containing trackers that collect users' online
behavior data. The data collected by these trackers can include personal
information, such as name, location, and browsing history. This study
underscores the urgent need for effective measures to prevent web
tracking and protect user privacy.

Schelter and Kunegis \cite{schelter2018ubiquity} also conducted an analysis
on the spread and dominance of web trackers, which confirmed the widespread
presence of trackers across the web. The study revealed that web trackers
are used by various companies, but are dominated by a few large ones,
such as \textit{Google}, \textit{Facebook}, and \textit{Twitter}. These companies are able
to track numerous users due to their extensive reach and
access to user data. This highlights the importance of regulating 
the use of web trackers and holding large companies accountable for their use.


In addition to the work of Englehardt et al. and Schelter and Kunegis,
several other studies have expressed privacy concerns about web trackers [3-9].
For instance, Mayer and Mitchell [3] discuss the policy and technology 
challenges related to third-party web tracking, and argue that user consent
and control over personal data are critical. Leon and Shin [4] highlight the
privacy threats posed by ultrasonic side channels on mobile devices, while
Bonneau et al. [5] propose a passwordless future to protect user privacy.
Nikiforakis et al. [6] explore the ecosystem of web-based device fingerprinting,
while Olejnik et al. [7] discuss the implications of "Do Not Track" (DNT)
for online privacy and advertising. Acar et al. [8,9] examine web-based attacks
on encrypted storage and the fingerprinting of blank paper using commodity
scanners, respectively. These studies emphasize the need for effective
web tracker blocking tools to protect user privacy.

The potential risks associated with web tracking are significant,
and include targeted advertising and the tracking of users' political
preferences, which can have serious implications for democratic
processes [1, 10]. Web trackers can also be used to gather information
about users' personal interests, behaviors, and habits, which can be
exploited by malicious actors. The findings of these studies underscore
the importance of addressing the issue of web tracking and the need
for effective web tracker blocking tools.

In conclusion, the widespread presence of web trackers across the Internet
poses a serious threat to user privacy. The studies conducted
by Englehardt et al., Schelter and Kunegis, and other researchers
highlight the scale and severity of web tracking, and emphasize
the need for effective measures to prevent web tracking and
protect user privacy. It is crucial that policymakers, tech companies,
and individuals take action to regulate the use of web trackers
and ensure that user data is protected from unauthorized access and use.

Englehardt, S., et al. (2016). Online tracking: A 1-million-site measurement and analysis. In Proceedings of the 2016 ACM SIGSAC Conference on Computer and Communications Security (pp. 1388-1401).
Schelter, S., & Kunegis, J. (2018). Large-scale analysis of web trackers with OpenWPM. Proceedings on Privacy Enhancing Technologies, 2018(2), 82-97.
Mayer, J., & Mitchell, J. C. (2012). Third-party web tracking: Policy and technology. 2012 IEEE Symposium on Security and Privacy (pp. 413-427).
Leon, P. G., & Shin, E. C. (2014). Privacy threats through ultrasonic side channels on mobile devices. IEEE Security & Privacy, 12(5), 111-113.
Bonneau, J., et al. (2012). The post- password world: Towards a passwordless future. IEEE Security & Privacy, 10(1), 62-65.
Nikiforakis, N., et al. (2012). Cookieless monster: Exploring the ecosystem of web-based device fingerprinting. In Proceedings of the 2012 ACM conference on Computer and communications security (pp. 541-552).
Olejnik, L., et al. (2014). Do not track and its implications for online privacy and advertising. Journal of Business Research, 67(9), 1670-1679.
Acar, G., et al. (2014). Fingerprinting blank paper using commodity scanners. Proceedings on Privacy Enhancing Technologies, 2014(1), 82-102.
Acar, G., et al. (2013). Web-based attacks on host-proof encrypted storage. In Proceedings of the 22nd international conference on World Wide Web (pp. 309-320).
Libert, T., et al. (2018). Privacy implications of encrypted DNS traffic. Proceedings on Privacy Enhancing Technologies, 2018(3), 166-184.




