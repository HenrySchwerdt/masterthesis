\chapter{DevOps}
\label{cha:relatedwork}

In diesem Kapitel wird auf die verschiedenen Dimensionen des DevOps Begriffs eingegangen. Erst werden diese Faktoren anhand des CAMS Modells erläutert und danach wird näher auf die Automatisierung eingegangen. Dabei wird genauer allem auf die Continuous-Integration und Continuous-Deployment eingegangen.

\section{Definition}
DevOps lässt sich definieren als „DevOps is a set of practices that is trying to bridge developer-operations gap at the core of things
and at the same time covers all the aspects which help in speedy, optimized and high quality software
delivery” \cite[p. 22]{nagpal2014literature}. Ein idealer Arbeitsablauf nach dem DevOps Paradigma wird in Figur-\ref{fig:devOps} dargestellt. In dieser Abbildung wird der Arbeitsablauf von den beiden Arbeitsbereichen als ein durchgängiger Schleifenlauf beschrieben. Es beschriebt also eine nahtlose Zusammenarbeit von Development und Operations.


\section{CAMS Modell}
Die Definition von DevOps ist sehr kompakt, aber was sagt sie konkret aus. Das CAMS-Modell beantwortet diese Frage sehr klar und präzise. Es ist ein von Damon Edwards und John Willis geschaffener Rahmen, um die Zusammenarbeit und Leistungsfähigkeit eines Unternehmens zu steigern und den Mitarbeitern anhand vier verschiedener Ziele den DevOps-Gedanken zu erläutern.
\subsection{Culture - Kultur}
Der erste Aspekt des CAMS Modells beschäftigt sich mit der Kultur. Auch wenn bei DevOps einige ziemlich fortschrittliche Technologien und Tools einsetzen werden, sind die Probleme, die zu lösen versucht werden, im Kern mit Menschen und Unternehmen verbunden. Die Kultur wird durch die Interaktion zwischen Menschen und Gruppen definiert und durch deren Verhalten bestimmt. Die Kommunikation kann erheblich verbessert werden, wenn ein gegenseitiges Verständnis für andere, deren Ziele und Verantwortlichkeiten vorhanden ist. Das betrifft im Kontext von DevOps hauptsächlich die Zusammenarbeit der Entwickler und der Operatoren, aber auch die generelle Kommunikation innerhalb des Teams \cite{aljundi2018tools, perera2017improve}. In den traditionellen Geschäftsmodellen der Informationstechnologie wurden Entwickler und Betrieb in zwei unterschiedliche Gruppen aufgeteilt. Sie sprachen im Wesentlichen unterschiedliche Sprachen, da die Entwickler für Innovation und Kreation zuständig waren, während das Betriebspersonal für die Aufrechterhaltung stabiler Umgebungen und Infrastrukturen verantwortlich war. Diese konkurrierenden Ziele führten häufig zu Reibereien, da jede Gruppe der anderen vorwarf, sie würde sie an der Erfüllung ihrer Aufgaben hindern. Ein wichtiges Ziel von DevOps ist es, die Unternehmenskultur zu ändern, indem die Verantwortung geteilt wird und die Teams an einem Strang ziehen \cite{perera2017improve}.

\subsection{Automation - Automatisierung}
Automatisierung ist ein weiterer wichtiger Aspekt von DevOps. Es ist nicht mehr nötig, vor jedem Release die Erlaubnis eines Administrators einzuholen. Deshalb können die Features ohne Hürden schneller live gehen. Automatisierung spart Zeit, Mühe und Geld. Wie bei der Kultur liegt der Schwerpunkt auf Menschen und Prozessen und nicht auf Tools. Wenn das Team die Kultur und die Ziele eines Unternehmens versteht, können diese durch das Bereitstellen von Infrastruktur als Code und des Einsatzes von Pipelines für Continuous-Integration und Continuous-Delivery verstärken werden. Es ist wichtig, die Automatisierung als einen Booster zu betrachten, der die Gesamtvorteile von DevOps steigert \cite{mohammad2018improve}. Jedoch sollte bei der Automatisierung auch auf die Auswahl der Tools geachtet werden, weil diese sehr zuverlässig sein müssen. Funktioniert ein Tool in der ganzen Pipeline nicht, dann können keine neuen Releases mehr auf das Produktionssystem gelangen.

\subsection{Measurement - Messung}
Anhand von Messungen lässt sich feststellen, ob Fortschritte in die gewünschte Richtung gemacht werden. Bei der Verwendung von Messwerten können zwei wichtige Probleme auftreten. Erstens: Es ist wichtig, dass die richtigen Messgrößen verwenden werden. Zweitens: Es müssen die richtigen Metriken ausgewählt werden. DevOps-Praktiken ermutigen Teams den gesamten Betrieb zu betrachten und ihn als Ganzes zu bewerten und sich nicht nur auf kleine Teile zu konzentrieren. Zu den wichtigsten Kennzahlen gehören unteranderem Einnahmen, Kosten, Umsatz, mittlere Wiederherstellungszeit, mittlere Zeit zwischen zwei Ausfällen und Mitarbeiterzufriedenheit \cite{mohammad2018improve}.

\subsection{Sharing - Teilen}
Bei DevOps-Prozessen, ähnlich wie bei agilen Pradigmen (z.B. Scrum), wird viel Wert auf Transparenz und Offenheit gelegt. Die Verbreitung von Wissen trägt dazu bei, die Feedbackschleifen zu straffen und ermöglicht es der Organisation, sich kontinuierlich zu verbessern. Diese kollektive Intelligenz macht das Team zu einer effizienteren Einheit und ermöglicht es ihm, mehr als nur die Summe seiner Teile zu sein.

\section{CI / CD}

Ein wichtiger Teil des DevOps-Prozesses besteht in der Automatisierung der Entwicklungspipeline. Dabei gibt es drei verschiedene Prozesse, mit wessen Hilfe es möglich ist alle Schritte von einer Änderung in der Codebasis bis zu einem Deployment auf die Liveumgebung automatisieren zu lassen.
Figur beschreibt so eine Pipeline und zeigt auf, welche Schritte zu welchem Paradigma gehören. Dort ist zu sehen, dass bei einer Änderung in der Codebasis im Rahmen der Continuous-Integration ein Build angestoßen wird und danach Unit-Tests ausgeführt werden. Sollten diese positiv verlaufen, wird in der Continuous-Delivery das System auf eine Entwicklungsumgebung deployed. Anschließend werden die Komponenten des Systems untereinander im Rahmen von Acceptance oder Smoke-Tests  getestet. Im letzten Schritt wird durch das Continuous-Deployment der Stand auf die Produktionsumgebung gelassen. Diese hier beschriebene Pipeline steht jedoch nur exemplarisch für eine CI / CD Pipeline, weil es in jeden Anwendungsfall eine unterschiedliche Reihung der Schritte gibt. Jedoch darf bei allen Pipelines der Code und der Build nicht veränderbar sein, damit immer ein funktionierendes System live gestellt wird. 

\subsection{Continuous-Integration}
Kontinuierliche Integration (Continuous Integration, CI) ist ein bekannter Softwareentwicklungsprozess \cite{fitzgerald2017continuous}, bei dem die Mitglieder eines Teams die Entwicklungsarbeit (z.B. Code) häufig integrieren und zusammenführen (z.B. viele Male pro Tag). Softwareunternehmen nutzen CI, um einen kürzeren und häufigeren Release-Zyklus zu erreichen, die Softwarequalität zu verbessern und die Produktivität ihrer Teams zu steigern \cite{fitzgerald2017continuous}. Diese Methode umfasst die Erstellung und das Testen von automatisierter Software \cite{leppanen2015highways, shahin2017continuous}.

\subsection{Continuous-Delivery}
Mit Continuous Delivery (CDE) soll sichergestellt werden, dass eine Anwendung immer produktionsbereit ist, nachdem sie automatisierte Tests und Qualitätskontrollen bestanden hat \cite{weber2016developing, humble2010continuous}.
CDE nutzt eine Reihe von Prozessen, darunter die kontinuierliche Integration und die Automatisierung der Bereitstellung, um Software automatisch in einer produktionsähnlichen Umgebung bereitzustellen \cite{shahin2017beyond}. Laut \cite{humble2010continuous, shahin2017beyond} hat diese Methode eine Reihe von Vorteilen, darunter ein geringeres Bereitstellungsrisiko, niedrigere Kosten und schnelleres Benutzerfeedback \cite{shahin2017continuous}.

\subsection{Continuous-Deployment}
Die Methode des Continuous Deployment (CD) geht noch einen Schritt weiter und stellt die Anwendung automatisch und kontinuierlich in der Produktions- oder Liveumgebung bereit \cite{weber2016developing}. Der Unterscheid zwischen Continuous Deployment und Continuous Delivery wird in akademischen und industriellen Kreisen heftig diskutiert \cite{fitzgerald2017continuous, weber2016developing, humble2010continuous}. Das releasen auf einer Produktionsumgebung (d.h. tatsächliche Kunden) unterscheidet Continuous Deployment von Continuous Delivery: Der Zweck der Continuous-Deployment-Praxis besteht darin, jede Änderung automatisch und kontinuierlich in der Produktionsumgebung bereitzustellen. Es ist erwähnenswert, dass die CD-Praxis die CDE-Praxis impliziert, aber nicht umgekehrt. Während das endgültige Deployment bei CDE ein manueller Schritt ist, sollte es bei CD keine manuellen Schritte geben, da die Änderungen über eine Deployment-Pipeline in der Produktion bereitgestellt werden, sobald die Entwickler sie festschreiben. CDE ist eine Pull-basierte Strategie, bei der ein Unternehmen bestimmt, was und wann es bereitgestellt wird; CD ist eine Push-basierte Methode \cite{skelton2016continuous}. Anders ausgedrückt: Der Anwendungsbereich von CDE umfasst keine häufigen und automatisierten Releases, daher ist CD eine Fortsetzung von CDE. Während die CDE-Praxis auf alle Arten von Systemen und Organisationen angewendet werden kann, ist die CD-Praxis möglicherweise nur für bestimmte Arten von Organisationen oder Systemen geeignet \cite{humble2010continuous, skelton2016continuous, shahin2017continuous}.



