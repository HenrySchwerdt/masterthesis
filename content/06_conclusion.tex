\chapter{Conclusion}
\label{cha:conclusion}
Throughout this thesis, we have embarked on a comprehensive exploration of the development and implementation of our web extension,
aimed at augmenting user privacy by blocking tracking requests. Our primary focus has been on the utilization of machine learning
models, specifically Model 1 and Model 2, to identify and intercept web trackers in real-time. Through extensive evaluations
and analyses, we have gained valuable insights into the performance and effectiveness of these models.

The training and evaluation of our models involved the crawl data of Raschke et al. \cite{raschke_philip_2022_7123945}. The training and test results
have been highly encouraging, as evidenced by the
evaluation metrics. Model 1 demonstrated an impressive precision of 0.98, a recall of 0.99, and an F1-score of 0.98 for classifying
non-trackers, while achieving a precision of 0.97, a recall of 0.96, and an F1-score of 0.96 for classifying trackers. On the other
hand, Model 2 exhibited a precision of 0.99, a recall of 0.98, and an F1-score of 0.98 for non-trackers, along with a precision of 0.95,
a recall of 0.97, and an F1-score of 0.96 for trackers. These evaluation values affirm the robustness and efficacy of our machine
learning models in accurately classifying web trackers and non-trackers.

Moreover, we conducted a comparative analysis by crawling the Tranco Top 1K websites and collecting the requests. The performance of our models and 
the renowned Brave browser, known for its
built-in advertisement and tracking blocking capabilities. Surprisingly, our machine learning models demonstrated comparable
results in blocking trackers. When comparing the effort it takes to create these TPLs and the implementation of a performant 
rule engine to block tracking requests in real-time, we can see that the ML approach offers a dominant position. The ML models can be trained in ten minutes
and used in the extension immediately. Furthermore, the creation of new models is incredible simple and only requires browsing data for training.
This outcome emphasizes the effectiveness and potential of our models in real-time tracker interception,
further highlighting their relevance in preserving user privacy during web browsing.

Nevertheless, it is essential to acknowledge and address the limitations of our web extension to ensure its optimal performance.
One such limitation pertains to the scope of our web crawls. While our utilization of the Tranco Top 1K websites provided a solid
foundation, expanding the crawl dataset to encompass a broader range of websites and simulating user interactions would yield more
comprehensive insights into the extension's tracking blocking capabilities.

Additionally, in order to gain a deeper understanding of user experience and potential site breakage issues, conducting user studies
becomes imperative. Such studies would allow us to evaluate the performance of our models from a usability standpoint and assess any
adverse effects on website functionality. Furthermore, conducting performance assessments, particularly in relation to page loading
times, would provide valuable insights to optimize the extension's efficiency.

To summarize, our web extension represents a significant advancement in safeguarding user privacy by effectively blocking tracking
requests with AI. The evaluation of our models, along with the comparative analysis against the Brave browser, showcases their remarkable
performance in reducing the presence of trackers. Nonetheless, we recognize the need for continuous evaluation, refinement, and
improvement. By expanding the web crawl dataset, conducting user studies, and optimizing performance, we aim to further enhance
the functionality of our extension and address its limitations. Moreover, there is a great need for training data which represents 
the ground truth of web tracking. For that we implemented our backend to support the creation of one such dataset.

As we move forward, we remain committed to developing and refining our web extension, incorporating user feedback, and conducting
rigorous evaluations. Our ultimate goal is to provide users with a safer and more privacy-focused browsing experience. The successful
integration of machine learning models in real-time tracker blocking demonstrated by our extension underscores the potential of this
technology in combating web tracking. In the future we plan on creating different training strategies, ones leveraging supervised
learning and ones that utilize reinforcement learning. In addition, we plan on publishing our backend to the web and make it accessible 
for anyone who wants to contribute. There it will be possible to create own models and train them all against the same dataset to ensure 
comparability between the different ML solutions. We are dedicated to the ongoing enhancement of our extension, leveraging machine learning to
ensure the highest level of privacy and to create a common ground truth for training new models.

