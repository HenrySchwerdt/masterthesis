\chapter{Ausblick}
\label{cha:conclusion}

Das DevOps Paradigma ist in der Wirtschaft angekommen. Die meisten großen Softwareunternehmen setzten auf agile Teams aus den Bereichen Development und Operations. Dabei organisieren sich diese Unternehmen unterschiedlich. Beispielsweise hat Spotify ein ganz eigenes System für ihre Softwareentwicklung durch verschieden Teams entwickelt \cite{spotify}. Zusammenfassend lässt sich aber sagen, dass die meisten Frameworks die Kerngedanken von DevOps nutzen oder erweitern. Diese Auswirkungen sorgen im Unternehmensalltag dafür, dass kombinierte Teams und funktionsübergreifende Teams sich größerer Beliebtheit erfreuen und öfter eingesetzt werden. CI / CD spielen in diesem Rahmen auch eine große Rolle. In vielen Cloud-basierten Unternehmen sind die Pipelines zu Großteilen oder komplett automatisiert. Es werden immer mehr Tools implementiert, welche die Automatisierung erleichtern sollen. Dieses Entwicklung erleichtern die migration auf einen CI / CD Ansatz und verstärkt die Motivation DevOps als Paradigma in verschiedenen Teams einzusetzen. 

Eines der Hindernisse für die Verbreitung der Automatisierung ist der Fachkräftemangel, denn um so ein System zu entwicklen braucht der Verantwortliche viel Verständnis über die neusten Technologien, Tools und die eigene Anwendung. Wenn aber ein solche System erfolgreich umgesetzt wird, führt das zu einem deutlich flexibleren und schnelleren Release-System. Es kann davon ausgegangen werden, dass DevOps die Zukunft der Softwareentwicklung ist. Das lässt sich unter anderem daran erkennen, dass die Performance, die Sicherheit mit SecOps und die Agilität durch den DevOps Ansatz verstärkt werden. Auch sind Unternehmen benachteiligt, wenn diese keine automatisierte Deployment-Pipeline aufweisen. Es macht also nur Sinn auf einen DevOps Ansatz zu setzten.