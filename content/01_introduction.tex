\chapter{Introduction}
\label{cha:introduction}

The evolution of the internet has revolutionized the way we interact with the world around us.
With HTML, CSS, and JavaScript, websites can offer a nearly infinite number of possibilities for user
engagement and interaction. However, with these new possibilities comes a vulnerability to well-known
security problems like Cross-Site Scripting (XSS), which can allow unauthorized third parties to access
and manipulate sensitive user data.

Web tracking is a ubiquitous phenomenon that poses significant threats to online privacy and security.
The vast amount of personal data being collected by various third-party trackers can be used to create
detailed profiles of individuals, including their browsing history, location, interests, and habits.
This information can be used for targeted advertising, but it can also be sold to other parties,
including those with malicious intent. Furthermore, tracking can enable identity theft,
fraud, and other forms of cybercrime. In addition to the threats to individual privacy and security,
web tracking can also have broader societal implications. The accumulation of vast amounts of data by a
few companies can give them unprecedented power to influence public opinion and behavior.
Additionally, the lack of transparency and accountability in the web tracking industry raises
concerns about democratic governance, as it allows powerful entities to operate with
little oversight. As such, web tracking represents a significant challenge for policymakers,
technologists, and individuals seeking to protect online privacy and security.

Third-party services play an important role in web development by providing various functionalities
such as authentication, CSS styling libraries, target advertisement, and content APIs. 
However, these services can also collect data without the end user's consent or knowledge,
which can be used to track browsing behavior and profit from this data. 
While not all third-party services are malicious, it can be challenging for users to distinguish between
legitimate and potentially harmful services.

To address this challenge, many researchers are working on automated solutions 
for tracker identification, using machine learning algorithms trained on data generated
by internet crawls. However, these methods can be limited by their reliance
on crowdsourced blocking lists, which can quickly become outdated in
the rapidly evolving landscape of the internet.

To contribute to this research and educate end users, I have developed a Chrome web extension
that uses a neural network model to detect tracking requests in real-time. 
The extension visualizes web traffic generated by the user in a directed graph,
allowing users to analyze their browsing behavior and identify potential privacy concerns.
This tool also offers additional features for analyzing the graph and determining metrics such as individual node in and out-degree.

In this paper, I will describe the design and implementation of the web extension,
discuss the results of experiments conducted to evaluate its performance,
and provide insights into potential future research directions.
Specifically, Chapter 2 will provide an overview of the current state of research
in the field of web tracking detection, while Chapter 3 will describe the design decisions
and considerations that went into the creation of the extension. 
Chapter 4 will outline the implementation details of the extension and
discuss how various libraries were integrated to optimize its performance.
Chapter 5 will present the results of experiments conducted to evaluate
the performance of the extension, and Chapter 6 will conclude the paper
with a discussion of future research directions and potential improvements to the extension.



